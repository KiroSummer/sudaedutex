%    This program is free software: you can redistribute it and/or modify
%    it under the terms of the GNU General Public License as published by
%    the Free Software Foundation, either version 3 of the License, or
%    (at your option) any later version.
%
%    This program is distributed in the hope that it will be useful,
%    but WITHOUT ANY WARRANTY; without even the implied warranty of
%    MERCHANTABILITY or FITNESS FOR A PARTICULAR PURPOSE.  See the
%    GNU General Public License for more details.
%
%    You should have received a copy of the GNU General Public License
%    along with this program.  If not, see <http://www.gnu.org/licenses/>.

%========================================
%	苏州大学本科生论文LaTeX模板
%	2010年 05月 29日 星期六 19:47:22 CST
%	By telive
%	Email:	tellive@gmail.com
%========================================

%========================================
%		Packages used in this template
\usepackage{xeCJK}			% 中文支持
\usepackage{pdfpages}		% 插入pdf
\usepackage{graphicx}		% 图形支持
\usepackage{amssymb}		% 数学符号
\usepackage{amsmath}		% 数学符号
\usepackage{fancybox}		% 方框文字
\usepackage{wrapfig}		% 图片文字环绕
\usepackage{fancyhdr}		% 页眉设置
\usepackage{cite}			% 引用文献
\usepackage{indentfirst}	% 首行缩进 
\usepackage[colorlinks,linkcolor=blue,citecolor=blue]{hyperref}		% 让tableofcontents支持超链接
\usepackage[top=1in,bottom=1in,left=1.4in,right=1.2in]{geometry}	% 设置页边距
%\usepackage[top=0.8in,bottom=0.8in,left=1.2in,right=0.6in]{geometry}  %设置页边距(学校的要求)

%========================================
%		Settings
\setCJKmainfont{SimSun}		% 字体设置,宋体
\setlength{\parindent}{2em}	% 首行缩进,2字符
%\setCJKmainfont{文泉驿正黑}

%========================================
%		Redefine commands 
\renewcommand\abstractname{摘\ 要}		% 摘要 ,
\renewcommand{\figurename}{图} 			% 图
\renewcommand\refname{参考文献}			% 参考文献
\renewcommand\contentsname{\centerline{目录}}	%目录居中
\renewcommand{\today}{\number\year 年 \number\month 月 \number\day 日}	%中文日期

%========================================
%		Header Settings
\pagestyle{fancy}			% 
\chead{ \bfseries 苏州大学本科生毕业设计(论文)}	% 页眉中部
\lhead{}		% 页眉左部,设为空
\rhead{}		% 页眉右部,设为空

%========================================

